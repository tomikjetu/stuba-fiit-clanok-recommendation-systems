\section{Two Tower Neural Networks} \label{nn}

Above the input layer is the embedding layer; it is a fully
connected layer that projects the sparse representation to
a dense vector.  \cite{10.1145/3038912.3052569} \cite{DBLP:journals/corr/abs-1708-05027}    

$\begin{bmatrix}
1 & 2 & 3\\
a & b & c
\end{bmatrix} $ is a matrix

\subsection{Cold Start}\label{cold-start}

When a new user registers on a social platform, the platform doesn't have any history about the user. Hence, it cannot personalize user's timeline. One option could be to initialize the user vector randomly, or find a similar vector to the most popular ones.

An option for creating embeddings of new posts would be analyzing the content features like the photo, description and hastags.

To improve the precision of recommendation of new items, here are a few proposed algorithms. 

\begin{comment}
\noindent\rule{2cm}{0.4pt}

aTwo tower neural network architecture 
User embedding
Post embedding
embedding = vector
normalize vector
dot product of post and user vector
pre trained embeddings

\noindent\rule{2cm}{0.4pt}
\end{comment}

\section{Convolutional Neural Networks}\label{cnn}

Most large corporations remain closed-source, keeping competition on the internet. With a few exceptions, with the the recent change of X (Twitter) going open-source. The next following subsections will explain some of the popular open-source image recognition neural networks and how it works in general.


\begin{comment}
\subsection{Alex Net}

\url{https://github.com/amir-saniyan/AlexNet}

\url{https://proceedings.neurips.cc/paper_files/paper/2012/file/c399862d3b9d6b76c8436e924a68c45b-Paper.pdf}

\subsection{VGGNet}

\url{https://github.com/deepblacksky/VGG_tensorflow?tab=readme-ov-file} 

\url{https://arxiv.org/abs/1409.1556}
\subsection{ResNet}

\subsection{GoogLeNet}
GoogLeNet (Inception)

\url{https://github.com/conan7882/GoogLeNet-Inception}

\url{https://research.google/pubs/going-deeper-with-convolutions/}
\subsection{EfficientNet}

\url{https://arxiv.org/abs/1905.11946}

\url{https://github.com/qubvel/efficientnet}
\subsection{Siamese Networks}

\url{https://www.cs.cmu.edu/~rsalakhu/papers/oneshot1.pdf}

\subsection{X (Twitter)}\label{cnn/xalgorithm}

\url{https://github.com/twitter/the-algorithm}

\url{https://blog.x.com/engineering/en_us/topics/open-source/2023/twitter-recommendation-algorithm}

RAPID NET

\end{comment}